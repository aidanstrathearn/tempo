\documentclass[a4paper, aps, pra,twocolumn]{revtex4-1}

\usepackage[english]{babel}
%\usepackage[utf8]{inputenc}
\usepackage{amsmath}
\usepackage{mathtools}
\usepackage{graphicx}
\usepackage{hyperref}
\usepackage{braket}
\usepackage[colorinlistoftodos]{todonotes}

\newcommand{\as}[1]{{\color{green}{#1}}}
\newcommand{\bl}[1]{{\color{red}{#1}}}
\newcommand{\pk}[1]{{\color{blue}{#1}}}

\newcommand{\dket}[1]{\left.\Ket{#1}\right\rangle}
\newcommand{\dbra}[1]{\left\langle\Bra{#1}\right.}
\newcommand{\dbraket}[1]{\left\langle\Braket{#1}\right\rangle}
\newcommand{\rSA}{\rho_R^A}

\let\Re\relax \DeclareMathOperator\Re{\mathrm{Re}}%
\let\Im\relax \DeclareMathOperator\Im{\mathrm{Im}}%
\DeclareMathOperator\Tr{\mathrm{Tr}}%
\DeclareMathOperator\sinc{\mathrm{sinc}}%


\begin{document}
\title{Tensor Network Representation of Influence Functional}

\author{A.~Strathearn}

\date{\today}

\begin{abstract}


\end{abstract}

\maketitle



\section{Discretized Path Integral}
\label{sec:scheme}

The generic Hamiltonian of such models is
\begin{align}
\label{eq:hamil}
 H=&H_0+ \hat{s}\sum_\alpha \hat{B}_\alpha + \sum_\alpha \omega_\alpha a^\dagger_\alpha a_\alpha \\
=&H_0+H_B,
 \end{align}
where $H_0$ is the free system Hamiltonian and $H_B$ contains both the free bath Hamiltonian and the system-bath interaction. Here,  $a_\alpha^\dagger$ ($a_\alpha$) and $\omega_\alpha$ are the creation (annihilation) operators and frequencies of the $\alpha$th oscillator. The system operator $\hat{s}$ couples to the bath operators $\hat{B}_\alpha=g_\alpha a_\alpha+g_\alpha^* a_\alpha^\dagger$ with coupling constants $g_\alpha$.

To simplify our notation we work in the Liouvillian representation such that operators in Hilbert space are represented by vectors in Liouville space. To parameterise the $D$ dimensional Hilbert space of the system we use the $D$ eigenstates of $\hat{s}$. Operators in this space are vectorized in the following way
\begin{align}
\hat{\rho}_R&=\sum_{s^+, s^-}\rho_{s^+ s^-}\ket{s^+}\bra{s^-} \nonumber\\
&\equiv\sum_{S} \rho_S\dket{S}\equiv\dket{\rho_R},
\end{align}
where the sum over $S$ runs over the $D^2$ pairs of $\{s^+,s^-\}$ and $\hat{s}\ket{s^+}=s^+\ket{s^+}$ and likewise for $s^-$. We use the notation $\dket{x}$ to mean a vector in Liouville space.
The bath Hilbert space can also be represented in a similar way, though we do not need to define its basis explicitly in what follows.
The evolution of the reduced system, assuming factorizing initial conditions is now represented as:
\begin{align}
\label{redrho}
\dket{\rho_R(t)}=\Tr_B\left[\text{e}^{\mathcal{L} t}\dket{\rho_R(0)}\dket{\rho_B}\right],
\end{align}
with the Liouvillian $\mathcal{L}=\mathcal{L}_0+\mathcal{L}_B$, where $\mathcal{L}_0$ and $\mathcal{L}_B$ generate coherent evolution caused by $H_0$ and $H_B$ respectively.
In addition to factorising initial conditions we also assume the initial state of the bath is that of thermal equilibrium when no system is present $\rho_B=\exp(-\sum_\alpha \omega_\alpha a^\dagger_\alpha a_\alpha/T)/\mathcal{Z}$, with  temperature $T$ and partition function $\mathcal{Z}$.


The first approximation made to make  Eq.~\eqref{redrho} computable is to factorize the long time propagator into $N$ short time propagators $\text{e}^{\mathcal{L} t}=(\text{e}^{\mathcal{L} \Delta t})^N$ and then to employ a Trotter splitting between the system and bath parts~\cite{trotter1959}
\begin{align}
\label{trot}
\text{e}^{\mathcal{L} \Delta t}\approx \text{e}^{\mathcal{L}_B \Delta t}\text{e}^{\mathcal{L}_0 \Delta t},
\end{align}
on each of these. The error introduced in this process is $\mathcal{O}(\Delta t^2)$.
We note that the argument that now follows can be easily adapted to use a symmetrized Trotter splitting~\cite{suzuki1976,makri_makarov_1995_i,makri_makarov_1995_ii} that improves the error to $\mathcal{O}(\Delta t^3)$. All the numerical results we present do include this symmetrized splitting, but for simplicity of notation we will use the definition in Eq.~\ref{trot} here.

Tracing out the bath degrees of freedom then results in a reduced density matrix at time $t_N=N\Delta t$, whose elements are 
\begin{multline}
\label{discreteevo}
\dbraket{S_N|\rho_R(t_N)}=\sum_{S_1 \ldots S_{N-1}}\left(\prod_{j=1}^N \prod_{j'=1}^j I_{j-j'}(S_j,S_{j'})\right)\\
\times \dbraket{S_1 |\rho_R(\Delta t)} .
\end{multline}
where
\begin{equation}
I_{j-j'}(S,S') = \begin{dcases*}
e^{- \phi_{j-j'}(S,S')} &\text{ $j-j'\ne 1$}\\ 
\dbra{S}e^{\mathcal{L}_0 \Delta t}\dket{S'}e^{- \phi_{j-j'}(S,S')} &\text{ $j-j'=1$}
\end{dcases*},
\end{equation}
with
\begin{equation}
 \phi_{j-j'}(S,S')=(s^+-s^-)({s'}^+\eta_{j-j'}-{s'}^-\eta_{j-j'}^*).
\end{equation}
and 
\begin{equation}
\dbra{S}e^{\mathcal{L}_0 \Delta t}\dket{S'}= \bra{s^+}\text{e}^{-iH_0 \Delta t}\ket{{s'}^+}\bra{{s'}^-}\text{e}^{iH_0 \Delta t}\ket{s^-}
\end{equation}


The coefficients $\eta_{k-k'}$ quantify the non-Markovian `interaction' between the reduced system at different times $t_k$ and $t_{k'}$ and are defined as
\begin{equation}
\eta_{k-k'} = \begin{dcases*}
\int_{t_{k-1}}^{t_k}\int_{t_{k'-1}}^{t_{k'}}C(t'-t'')dt''dt' &\text{ $k\ne k'$}\\ 
\int_{t_{k-1}}^{t_k}\int_{t_{k-1}}^{t'}C(t'-t'')dt''dt' &\text{ $k=k'$}
\end{dcases*},
\end{equation}
in terms of the bath autocorrelation function 
\begin{align}\label{bathcorr}
C(t)&=\sum_\alpha \braket{\hat{B}_\alpha(t+s)\hat{B}_\alpha(s)}\\
&=\int_0^\infty d\omega J(\omega)(\coth( \omega/2 T)\cos(\omega t)-i \sin(\omega t)),
\end{align}
where $J(\omega)=\sum_\alpha |g_\alpha|^2 \delta(\omega_\alpha-\omega)$ is the spectral density of the bath.
Note in the 1st order trotter splitting we use here the bath lags the system by a timestep $\Delta t$ so the system has been propagated freely for this amount of time in EQ.

\section{Tensor Network}
We interpret the summand of the discretized path integral equation \eqref{iffac} as the components of a rank-n tensor
\begin{equation} \label{inftens}
 T_{S_N, S_{N-1} \ldots S_1}=\left(\prod_{j=1}^N \prod_{j'=1}^j I_{j-j'}(S_j,S_{j'})\right)\\
\times \dbraket{S_1 |\rho_R(\Delta t)},
\end{equation}
of which there are $D^N$, where $D=d^2$ is the dimension of the Lioville space and $d$ that of the underlying Hilbert space.
We will show that $T_{S_N, S_{N-1} \ldots S_1}$ can be written as a tensor network consisting of $N(N+1)/2$ tensors with rank-4 at most and that a TEBD-type method can be used to contract this network, with storage requirements now $O(N D^2 d_b^2$ where $d_b$ is the chosen bond dimension.

First, gather up terms in inner piece of double product in \eqref{iffac} into a single object, which we write as components of a rank-j tensor $\mathbf{\Lambda}_j$
\begin{align}\label{lamdef}
 (\mathbf{\Lambda}_j)_{S_j, S_{j-1} \ldots S_{1}}&= \prod_{j'=1}^{j} I_{j-j'}(S_j,S_{j'}) \\
 &= \prod_{\Delta k=0}^{j-1} I_{\Delta k}(S_j,S_{j-\Delta k}).
\end{align}
In the remaining product over the $\mathbf{\Lambda_j}$ any pair of adjacent terms in the product can be written in terms of a sum.  Using the einstein summation convention that a repeated index appearing once as a superscript and once as a subscript are summed over, the $j$-th pair of terms in product $\prod_{j=1}^{N}(\mathbf{\Lambda}_j)_{S_j, S_{j-1} \ldots S_{1}}$ can be written as
\begin{multline}
 (\mathbf{\Lambda}_j)_{S_{j}, S_{j-1}, \ldots S_{1}} (\mathbf{\Lambda}_{j-1})_{ S_{j-1} \ldots S_{1}} = \\
  (\tilde{\mathbf{\Lambda}}_j)_{ S_j, S_{j-1}, \ldots S_{1}}^{S'_{j-1} \ldots S'_{1}} (\mathbf{\Lambda}_{j-1})_{ S'_{j-1} \ldots S'_{1}}
\end{multline}
where we have defined the new rank-$(2j-1)$ tensor $\tilde{\mathbf{\Lambda}}_j$ with components
\begin{multline}
  (\tilde{\mathbf{\Lambda}}_j)_{S_j, S_{j-1}, \ldots S_{1}}^{ S'_{j-1} \ldots S'_{1}}=\left(\prod_{j'=1}^{j-1} \delta_{S_{j'}}^{S'_{j'}}\right) (\mathbf{\Lambda}_j)_{ S_j, S_{j-1}, \ldots S_{1}}\\
  = I_0(S_{j},S_{j}) \prod_{\Delta k=1}^{j-1} \delta_{S_{j-\Delta k}}^{S'_{j-\Delta k}} I_{\Delta k}(S_{j},S_{j-\Delta k})
\end{multline}
and the deltas satisfy
\begin{equation}
R_{... b ...}= \delta^a_b R_{... a ...}.
\end{equation}

Going through $\prod_{j=1}^{N}(\mathbf{\Lambda}_j)_{S_j, S_{j-1} \ldots S_{1}}$ replacing each $\mathbf{\Lambda}_j$ with a $\tilde{\mathbf{\Lambda}}_j$ thus allows us to write it as a tensor network with structure in FIG. 

Note that for each $\tilde{\mathbf{\Lambda}}_j$ we can combine all the upper indices and lower indices each into a single index to give a rank-2 tensor, or rectangular matrix, such that contractions between consecutive $\tilde{\mathbf{\Lambda}}_j$ can be written simply as matrix multiplication. The only $\tilde{\mathbf{\Lambda}}_j$ that doesn't have both an upper and lower index is the rank-1 $\tilde{\mathbf{\Lambda}}_1$, which combined with the initial state gives the components of the  rank-1 vector which the first rectangular matrix, $\tilde{\mathbf{\Lambda}}_2$, acts upon
\begin{equation}
 (\tilde{\mathbf{\Lambda}}_1)_{S_1}(\dket{\rho_R(\Delta t)})_{S_1}=I_0(S_1,S_1)\dbraket{S_1 |\rho_R(\Delta t)}.
\end{equation}
We can thus define the square matrix
\begin{equation}
 (\tilde{\mathbf{\Lambda'}}_1)_{S_1}^S=\delta_{S_1}^S (\tilde{\mathbf{\Lambda}}_1)_{S_1}
\end{equation}
such that
\begin{equation}
 (\tilde{\mathbf{\Lambda'}}_1)_{S_1}^S(\dket{\rho_R(\Delta t)})_{S}=(\tilde{\mathbf{\Lambda}}_1)_{S_1}(\dket{\rho_R(\Delta t)})_{S_1}.
\end{equation}
With this, the tensor whose components are given by EQ can be written in the coordinate independent form
\begin{equation}
\dket{\rho_A(t=N\Delta t)}= \tilde{\mathbf{\Lambda}}_N \tilde{\mathbf{\Lambda}}_{N-1} \ldots \tilde{\mathbf{\Lambda}}_{2}\tilde{\mathbf{\Lambda'}}_{1}\dket{\rho_R(\Delta t)}.
\end{equation}
We can consider $\tilde{\mathbf{\Lambda}}_{k}$ a superoperator which acts to the right on an operator space which looks like $k-1$ copies of the Hilbert space of our physical reduced system, $\mathcal{H}_1 \otimes \mathcal{H}_2 \otimes \ldots \mathcal{H}_{k-1} $, and produces a vectorized operator in a space of $k$ copies. Thus we call $\dket{\rho_A(t=N\Delta t)} \in \mathcal{H}_1 \otimes \mathcal{H}_2 \otimes \ldots \mathcal{H}_{N}$ the augmented density tensor with
\begin{equation}
\dket{\rho_A(t=N\Delta t)}= \sum_{S_1...S_N} T_{S_N, S_{N-1} \ldots S_1 }\dket{S_1}\otimes \dket{S_2} \otimes \ldots \dket{S_N}.
\end{equation}
The analogy with with a density matrix is because it looks like the density matrix for a chain of interacting copies of the physical reduced system and because it is Hermitian and has trace equal to one.
Our aim is to take this analogy to a 1D chain of interacting systems seriously and apply known methods to find an MPS of the augmented density tensor.
To do this we first must find an MPO representation of the superoperators $\tilde{\mathbf{\Lambda}}_{k}$.

In the same way that we ``tensorised'' the product of $(\mathbf{\Lambda}_k)_{S_k S_{k-1} \ldots S_1}$ by inserting $\delta$'s and summations into the product, each term in product defining $\tilde{\mathbf{\Lambda}}_{k}$, EQ, can similarly be cast into the form of tensor components.
Defining the rank-4 tensor $\mathbf{I}_{\Delta k}$ with components
\begin{align}\label{rank4}
(\mathbf{I}_{\Delta k})_{\alpha S}^{\alpha' S'}= \delta_{\alpha}^{\alpha'}\delta_{S}^{S'} I_{\Delta k}(\alpha,S)
\end{align}
it is straightforward to verify that the components of $\tilde{\mathbf{\Lambda}}_{k}$ can be written as, continuing to use the convention that an index appearing once as an upper and once as lower is summed over,
\begin{multline}
(\tilde{\mathbf{\Lambda}}_j)_{ S_j, S_{j-1}, \ldots S_{1}}^{S'_{j-1} \ldots S'_{1}}= \\ \delta_{S'_j}^{\alpha_j}\left(\prod_{\Delta k=0}^{j-1} (\mathbf{I}_{\Delta k})_{\alpha_{j-\Delta k} S_{j-\Delta k}}^{\alpha_{j-\Delta k-1} S'_{j-\Delta k} }\right) (\mathbf{1})_{\alpha_0},
\end{multline}
where $(\mathbf{1})_{\alpha_0}$ is the rank-1 tensor whose components are all equal to 1 and is included, along with the $\delta$, ensure the correct rank of $\tilde{\mathbf{\Lambda}}_j$. Their action is only on two of the terms in the product,
\begin{multline}
\delta^{\alpha_j}_{S'_j}(\mathbf{I}_{0})_{\alpha_j S_j}^{\alpha_{j-1} S'_{j}}=(\mathbf{I}_{0})_{\alpha_j S_j}^{\alpha_{j-1} \alpha_{j}} =\delta_{S_j}^{\alpha_{j-1}}I_{0}(S_j,S_j).
\end{multline}
and
\begin{align}
(\mathbf{I}_{j-1})_{\alpha_1 S_1}^{\alpha_0 S'_j}(\mathbf{1})_{\alpha_0}=\sum_{\alpha_0}(\mathbf{I}_{j-1})_{\alpha_1 S_1}^{\alpha_0 S'_1}= \delta_{S_1}^{S'_1}I_{j-1}(\alpha_1,S_1).
\end{align}
This decomposition into rank-4 tensors is shown in FIG.

We can alternatively write the decomposition into rank-4 tensors as a more conventional looking product of square matrices by defining the vectors $\dbra{I_0(S)}$, $\dket{\mathbf{1}}$ and matrices $\tilde{\mathbf{I}}_{\Delta k}(S,S')$ (the $S$'s here are considered labels rather tensor indices) with components
\begin{align}
 \dbraket{I_0(S)|\alpha}&=\delta_{S}^{\alpha}I_{0}(S,S)\\
 \dbraket{\alpha|\mathbf{1}}&= (\mathbf{1})_\alpha\\
 \dbraket{\alpha|\tilde{\mathbf{I}}_{\Delta k}(S,S')|\alpha'} &= \delta_{\alpha}^{\alpha'}\delta_{S}^{S'} I_{\Delta k}(\alpha,S)
\end{align}
such that
\begin{multline}
(\tilde{\mathbf{\Lambda}}_j)_{ S_j, S_{j-1}, \ldots S_{1}}^{S'_{j-1} \ldots S'_{1}}= \\
\dbra{I_0(S_j)}\tilde{\mathbf{I}}_{1}(S_{j-1},S'_{j-1}) \ldots \tilde{\mathbf{I}}_{j}(S_1,S'_1)\dket{\mathbf{1}}.
\end{multline}
Thus we have successfully found an MPO representation of $\tilde{\mathbf{\Lambda}}_j$.

\end{document}